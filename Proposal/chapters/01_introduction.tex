\section*{Introduction and structure}

The aim of this project is to implement a range of acceleration
structures targeted towards voxel ray marching, and to compare the
effectiveness of these methods for a range of functions.
These functions include, but are not limited to, rendering, updating,
memory cost, and creation.
The set of possible acceleration structures includes octrees,
contrees, brick maps, flat grids and 3D textures.

The project will be written in C++, with vulkan as the rendering API.
Only a few libraries are known to be needed, including GLFW for windowing,
ImGUI for UI, slang for shader compilation, and GLM for maths. The
shader language of choice is slang, as this allows the shaders to be compiled
during runtime, allowing for hot reloading.

\subsection*{Core project}
The core element of the project involves implementing the ray marcher
for the main set of methods. The main 3 methods are the octree, bricks
and flat grids. This puts the contrees and 3D textures as extensions
as they are adaptations of
the octree and flat grid respectively. With these methods the main
aim is the compare the render time for various scenes, that range in
size and complexity.

An explanation of the main structures is below

\begin{itemize}
  \item Flat grids \\
    Each voxel is individual stored in a single array. DDA can be used to
    trace through the grid and check for collisions.
  \item Octrees \\
    A tree where each node has 8 children. In particular sparse octrees will be
    used allowing the tree to have a varying depth. The voxels themselves are
    represented by the leaf nodes
  \item Bricks \\
    A mix between octrees and flat arrays. Similar to the octree in
    being a tree structure,
    except each leaf node is composed of a `brick', an
    $n*n*n$ collection of voxels.
\end{itemize}

\subsection*{Extensions}
There are a few possible extensions for this project.
One of the options involves implementing Contrees and 3D textures.

\begin{itemize}
  \item 3D textures \\
    Similar to the flat grid except an image is now used to store the
    voxel information, leveraging the GPU's acceleration for accessing texels.
    Can also be expanded to be used as a brick.
  \item Contrees \\
    Similar to octrees except each node is composed of 64 children.
\end{itemize}

In addition, extensions to what is expected of each structure can be
included. This includes comparison in creating the data structures,
live modifying of the structures, and serialization for storage.

A more advanced extension includes splitting the rendering of the scene between
devices. A basic approach is rendering the scene on one device with
another receiving it, though
this can be further extended to partial rendering on each device.
This is a method
that can be utilized by cloud gaming to reduce the network overhead
and to improve latency.
