\section*{Work Plan}

% Camera movement
% Shader hot reload
% Imgui
% Flat grid
% Octree
% Brick rendering
% Contree
% 3D texture
% Events

% Modification
% Octree / Contree
% Brick / Grid / 3D texture

% Serialization
% Octree / Contree
% Brick / Grid / 3D texture

% Lighting?
% Shadows
% AO
% Lights

% Split rendering
% Server / Client
% Networking
% Rendering between clients

\begin{longtblr}{colspec={clXX}, hlines, row{1}={}}
  Sprint & Dates & Plan & Milestones \\
  1 & Oct 13 -- Oct 27 &
  - Project setup \newline
  - Basic Vulkan rendering to the screen via Compute shaders \newline
  - Shader hot reloading \newline
  - ImGui setup \newline
  - Camera movement \newline
  - Event system
  &
  - Able to render a sphere through compute shaders \newline
  - Reposition around the sphere via the camera \newline
  - Modify properties of the sphere via ImGui \newline
  - Modify the sphere from the shader without closing the window
  \\
  2 & Oct 27 -- Nov 10 &
  - Start implementing ray marching structures \newline
  - 3D grid structure \newline
  - Octree structure
  &
  - Render a procedural voxel scene with the 3D grid and octree \newline
  - Switch between structures within ImGui \newline
  - Structures can be generated
  \\
  3 & Nov 10 -- Nov 24 &
  - Continue with implementing basic methods \newline
  - Brickmap support \newline
  - Contree support
  &
  - Render a scene with Brickmaps and contrees
  \\
  4 & Nov 24 -- Dec 08 &
  - Additional structures \newline
  - 3D textures \newline
  - Modification for flat grid
  &
  - Render a scene with 3D textures
  - Flat grid structure can be modified
  \\
  5 & Dec 08 -- Dec 22 &
  - Modification for octree / contree \newline
  - Modification for brickmap
  &
  - Octree and brickmap can be modified \newline
  - Cubes and spheres of voxels can be added and removed \newline
  - Voxels in various colours can be added
  \\
  6 & Dec 22 -- Jan 05 &
  - Mesh to Voxels \newline
  - Collect data for comparison of structures
  &
  - Be able to convert meshes to voxels \newline
  - Load files for scenes \newline
  - Change scene during runtime \newline
  - All previous milestones met
  \\
  7 & Jan 05 -- Jan 19 &
  - Start split rendering \newline
  - Split software into client and server \newline
  - Establish connection between client and server
  &
  - Be able to load the application as either a server or client \newline
  - Connect the client to the server \newline
  - Data can be shared between server and client
  \\
  8 & Jan 19 -- Feb 02 &
  - Implement 3D grid for split rendering
  &
  - Data for 3D grid can be shared from server to client \newline
  - Data can be requested from server
  \\
  9 & Feb 02 -- Feb 16 &
  - Implement structures across devices \newline
  - Implement Octree/Contree for split rendering
  &
  - Octree can be shared between server and client \newline
  - Nodes of the octree can be loaded as needed from server
  \\
  10 & Feb 16 -- Mar 02 &
  - Octree addition \newline
  - Implement loading for tree nodes
  &
  - Octree can be modified on client and updated server side \newline
  \\
  11 & Mar 02 -- Mar 16 &
  - Octree across devices \newline
  - Allow for nodes to be individually loaded between devices
  &
  - Only visible children are loaded on the client \newline
  - Newly visible children requested from server
  \\
  12 & Mar 16 -- Mar 30 &
  - Refactor / extra time
  &
  - Project should be completed
  \\
  13 \rightarrow & Mar 30 -- May 15 &
  - Dissertation
  &
\end{longtblr}

By the end of Michaelmas the core project should be finished to a
sufficient degree. This leaves Lent to work on the more complicated extensions.
In particular lent will be focused on the split rendering, with an interest in
implementing the octree/contree as this structure, along with
selective inclusion
of the tree nodes that are out of level of detail range, and those that are
obscured would require more communication between the client and server, and so
a more interesting challenge.

There are also some additional tasks, that if time allows can be included.
Notably this may include better lighting calculations. These are not
planned for as the time requirement for
split rendering is not easy to judge. This is why the last sprint is left open,
either for additional tasks, or for continuing work on the previous tasks.
