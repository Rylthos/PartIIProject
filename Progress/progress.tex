\documentclass[12pt]{report}

\usepackage{import}
\usepackage{parskip}
\usepackage{tabularray}

\usepackage[backend=biber]{biblatex}

\usepackage[left=2cm, right=2cm, top=2cm, bottom=2cm]{geometry}


\begin{document}

\begin{center}
  \Large\textbf{Comparison of Voxel Ray Marching Acceleration Structures}

  \begin{tabular}{lr}
    Aaron Danton & agrd2@cam.ac.uk
  \end{tabular}

  \vspace{0.5em}

  \textbf{Supervisors}

  \begin{tabular}{lr}
    Day-to-Day & Dounia Hammou \\
    Marking &  Rafal Mantiuk \\
  \end{tabular}

  \vspace{0.5em}

  \textbf{Director of Studies} \\
  Matthew Ireland \\

  \vspace{2em}

\end{center}

All core objectives have been met, which notably includes:
implementation of the 5 acceleration structures (Grid, Texture, Octree, Contree, Brickmap);
shader hot-reloading when a source file is edited;
an ability to measure the performance metrics of those structures;
scenes and structures can be changed during the runtime;
modification for the grid, texture and brickmap.
With the implementation of modification, key-frame animation support was also
added.

To facilitate the use of larger models and key-frame animation, a separate
program was written to convert meshes to voxels, and then handle the
generation of the acceleration structures along with their serialization. This program also handles the per
frame difference information to support animations, along with
handling scaling of the input models so that the output structures
are of a reliable size.

These structures are serialized via protocol buffers, allowing the serialized
structures to be used cross platform, and to remove the requirement of
handling serialization of arbitrary data.

With the core project meeting all success criteria, work has begun on
the split rendering extension. The main goals for which are to evaluate the
structures for use in a split-rendering environment (across a server and client).
To begin, structure agnostic methods are being implemented,
which includes image imposter, environment mapping, and depth image techniques.
These methods provide an increasing amount of moveability from the client between received
frames, at the cost of increased amounts of computation on
the client side. Image Imposter involves sending only the rendered frame, environment
mapping involves sending a 3D cube map of the environment, and sending the depth image
allows for 3D image warping techniques to be implemented.
Implementing these methods will allow for a qualitative
comparison over the costs of the structures in a server, based on the resource
requirement of the GPU and the bandwidth.

To facilitate the communication between the server and client, quic is being
used as the networking protocol, to allow for both lossy and lossless
communication. Messages are structured using protocol buffers,
allowing for easy extensibility and reliable cross-platform communication.
Individual image frames are encoded using ffmpeg which can then
be sent over Datagram messages before being decoded on the client.

\end{document}
